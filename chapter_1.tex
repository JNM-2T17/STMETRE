%%%%%%%%%%%%%%%%%%%%%%%%%%%%%%%%%%%%%%%%%%%%%%%%%%%%%%%%%%%%%%%%%%%%%%%%%%%%%%%%%%%%%%%%%%%%%%%%%%%%%%
%
%   Filename    : chapter_1.tex 
%
%   Description : This file will contain your Research Description.
%                 
%%%%%%%%%%%%%%%%%%%%%%%%%%%%%%%%%%%%%%%%%%%%%%%%%%%%%%%%%%%%%%%%%%%%%%%%%%%%%%%%%%%%%%%%%%%%%%%%%%%%%%

\chapter{Research Description}
\label{sec:researchdesc}

This chapter is an overview of the research undertaken in the field of community detection in social networks. 
This chapter is divided into four sections which are the current state of the technology, research objectives, scope and limitations, and significance of the research.

\section{Overview of the Current State of Technology}
\label{sec:overview}

Social media has become much more prevalent in recent years. People can now participate in what is called microblogging, a way for people to share their thoughts, status, and opinions in short posts, like Twitter, where posts are limited to one-hundred and forty characters. \cite{Java:2007}. As such, these social media platforms are a prime opportunity to mine sentiments and to detect communities in the social network. 

Sentiment analysis is, in one of its forms, analyzing a textual post and determining a characteristic from it such as whether it is subjective or objective or whether it is positive or negative \cite{Deitrick:2013}. Based on these analyses and the topics the users post about, community detection is clustering multiple users into groups where users within the group are more similar than users from outside the group \cite{Tang:2010}.

Numerous studies on community detection have already been done. \citeA{Zhang:2012} defined features that can be used to identify similarity and aggregated these similarities to detect communities. \citeA{Lim:2012:1} performed an inverted version in which they first defined interests and based on these interests, sought to extract communities from the network. 

Individual opinions of one specific user towards another was also studied by combining sentiment analysis with network analysis \cite{West:2014}. In addition to these works, some visualizations have already been created such as SocialHelix by \citeA{Cao:2015} which depicts two sides of an argument as strands in a helix and their intersection defines events.

However, it is noticeable that most of these studies mostly involved Twitter. As far as the research done so far by the group goes, there has yet to be a community detection tool that integrates data from Facebook into the computation. In this research, the group aims to address the fact that such a system does not yet exist by developing a visualization tool for community detection using sentiment analysis on the social networks Facebook and Twitter.

\section{Research Objectives}
\label{sec:researchobjectives}

\subsection{General Objective}
\label{sec:generalobjective}

To produce a visualization of the detected communities on data found on Facebook and Twitter

\subsection{Specific Objectives}
\label{sec:specificobjectives}

\begin{enumerate}
	\item To determine the various techniques and algorithms in detecting communities;
	\item To determine the appropriate parameters to use in detecting the communities;
	\item To determine how to evaluate the correctness of the detected communities
\end{enumerate}

\section{Scope and Limitations of the Research}
\label{sec:scopelimitations}

In detecting communities, research is needed to identify the appropriate algorithms for clustering users into communities. The group’s research will be limited to review of algorithms we found in the review of related literature, including the Infomap algorithm, speaker-listener label propagation algorithm, Markov stability, clique percolation method, k-means clustering, and divisive hierarchical clustering.

To perform these community detection algorithms, it is necessary to identify which parameters indicate one user’s similarity to another. Research will be done to identify these parameters, specifically how to evaluate them based on the raw data. The research is limited to sentiment analysis and elements which can be extracted from a user’s post, which may include follow networks, hashtags, mentions, and retweets, which were mostly inspired from literature which focused on Twitter \cite{Deitrick:2013,Zhang:2012,Lim:2012:1}. As such, Facebook specific features such as membership in groups and event participation may also be considered.

After community detection, it is necessary to determine whether the detected communities are correct. As such, research will be done to find appropriate metrics in determining the accuracy of detected communities. This algorithms will include average mutual following links per user per community or FPUPC \cite{Zhang:2012}, modularity \cite{Deitrick:2013}, and clustering coefficient \cite{Lim:2012:1}.

\section{Significance of the Research}
\label{sec:significance}

Community detection is a topic that has been explored quite recently as social media networks grew more prevalent. As such, existing algorithms have already been reviewed and used in community detection and sentiment analysis. However, regarding the domain of existing applications, most of them only consider data from Twitter. This research will finally include Facebook in the scope of its community detection. This research can also contribute to the notion that community detection is a relevant field of study in this day and age.

This research can also be a very useful tool in the domains of viral marketing and political endorsement. This means that companies and governments may benefit from this research. Interested companies may use the result of this research to improve their sales and marketing. The government may use this to gauge public opinion on certain issues and to see analytics about which geographical areas have a particular opinion. 