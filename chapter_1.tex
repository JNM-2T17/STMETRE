%%%%%%%%%%%%%%%%%%%%%%%%%%%%%%%%%%%%%%%%%%%%%%%%%%%%%%%%%%%%%%%%%%%%%%%%%%%%%%%%%%%%%%%%%%%%%%%%%%%%%%
%
%   Filename    : chapter_1.tex 
%
%   Description : This file will contain your Research Description.
%                 
%%%%%%%%%%%%%%%%%%%%%%%%%%%%%%%%%%%%%%%%%%%%%%%%%%%%%%%%%%%%%%%%%%%%%%%%%%%%%%%%%%%%%%%%%%%%%%%%%%%%%%

\chapter{Research Description}
\label{sec:researchdesc}

This chapter is an overview of the research undertaken in the field of community detection in social networks. 
This chapter is divided into four sections which are the current state of the technology, research objectives, scope and limitations, and significance of the research.

\section{Overview of the Current State of Technology}
\label{sec:overview}

Social media has become much more prevalent in recent years. People can now participate in what is called microblogging, a way for people to share their thoughts, status, and opinions in short posts, like Twitter, where posts are limited to one-hundred and forty characters \cite{Java:2007}. As such, these social media platforms are a prime opportunity to mine sentiments and to detect communities in the social network. 

Community detection is clustering multiple users into groups where users within the group are more similar than users from outside the group \cite{Tang:2010}. Community detection is necessary because it observes the interaction of multiple users as opposed to common mining methods which may only deal with local predictions i.e., predictions based on a single node as opposed to predictions considering the entire network of users. 

Numerous studies on community detection have already been done. \citeA{Zhang:2012} defined features such as text content similarity, URL similarity, hashtag similarity, following similarity, and retweeting similarity that can be used to identify similarity between two nodes and aggregated these similarities to detect communities. Their software provided the listings of users in each community. \citeA{Lim:2012:1} performed an inverted version in which they first defined interests and based on these interests, sought to extract communities from the network by identifying users that follow the top six celebrities, users with more than 10,000 followers, relating to the given interest. Their software outputted the size and clustering coefficient of each community 

Individual opinions of one specific user towards another were also studied by \citeA{West:2014} combining sentiment analysis, using an L2-regularized logistic-regression classifier, with network analysis, inferring one user`s opinion of another by analyzing their common links with other users. Their output was the area under the curve of the receiver operating characteristic (AUC/ROC) and the precision-recall curves (AUC/negPR). 

In addition to these works, some visualizations have already been created such as SocialHelix by \citeA{Cao:2015}, which uses the temporal extent of social communities, topics or events discussed, and user responses to topics and events to classify users into two sides of the argument. They then depict the two sides of an argument as strands in a double helix and their intersections defines events.

However, it is noticeable that all of these studies only involved Twitter and mainly used only common Twitter features for similarity analysis such as following networks, hashtag frequency, and retweet networks. According to \citeA{McCarthy:2014}, Facebook has 1.3 billion monthly active users compared to Twitter's 271 million. In addition to this, Facebook has certain unique features such as group membership, events, and reactions, which could provide more similarity parameters for community detection. Given that the volume of studies on community detection in Twitter outweighs the volume of studies about Facebook, the proponents wish to include Facebook, in addition to Twitter, in the considerations to determine which algorithms and features provide more accurate communities.

\section{Research Objectives}
\label{sec:researchobjectives}

\subsection{General Objective}
\label{sec:generalobjective}

To produce a visualization of detected communities on data found on Facebook and Twitter.

\subsection{Specific Objectives}
\label{sec:specificobjectives}

\begin{enumerate}
	\item To build a corpus of social media data;
	\item To determine the various techniques and algorithms in detecting communities;
	\item To determine the parameters/features to be used in detecting the communities;
	\item To determine how to evaluate the correctness of the detected communities;
	\item To implement a tool for the visualization of detected communities using the gathered information
\end{enumerate}

\section{Scope and Limitations of the Research}
\label{sec:scopelimitations}

In order to perform a study on community detection, it is necessary to build a corpus of social media data. This research will cover searching for Application Programming Interfaces (API) that will allow extraction of data from Facebook and Twitter and then using these APIs to build a body of data where community detection can be performed on. Data will include posts, profile information, and network information such as following list, follower list, and group membership.

Different techniques have been used in community detection. Among these techniques are the Infomap algorithm and the speaker-listener label propagation algorithm \cite{Deitrick:2013}.  This research will consider algorithms found in the review of related literature, including the Markov stability model, clique percolation method, k-means clustering, and divisive hierarchical clustering.

Before the proponents implement selected community detection algorithms, it is necessary to identify which parameters/features/attributes indicate one user`s similarity to another. Inquiry will be done to identify these parameters, specifically how to extract them based on the raw data. The research will be limited to sentiment analysis and elements which can be extracted from a user’s post, which may include follow networks, hashtags, mentions, and retweets, which were mostly inspired from literature which focused on Twitter \cite{Deitrick:2013,Zhang:2012,Lim:2012:1}. As such, Facebook specific features such as membership in groups and event participation may also be considered.

After community detection, it is necessary to determine whether the detected communities are sensible. Inquiry will be done to find appropriate metrics in determining the accuracy of detected communities. These algorithms will include average mutual following links per user per community or FPUPC \cite{Zhang:2012}, modularity \cite{Deitrick:2013}, and clustering coefficient \cite{Lim:2012:1}.

After inquiring about multiple community detection algorithms and similarity parameters, it will be necessary to implement the selected algorithms in a working system. The community detection model will then be implemented and augmented by a visualization tool which will be created that will only consider communities detected from data gathered from Facebook and Twitter.

\section{Significance of the Research}
\label{sec:significance}

Community detection is already a widely researched topic in the field of computer science. Our study will contribute to that field by exploring a domain that is not a frequently explored in the field of community detection, Facebook. Since Facebook has a larger user base than Twitter and more features, our study may determine if Facebook would produce better communities than Twitter, which would influence future studies about community detection in social media.

Our study can also be a very useful tool in the domains of viral marketing and political endorsement. This means that companies and governments may benefit from this research. Interested companies may use the result of our study to improve their sales and marketing. The government may use our study to gauge public opinion on certain issues and to see analytics about which geographical areas have a particular opinion. 




