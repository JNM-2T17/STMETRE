%%%%%%%%%%%%%%%%%%%%%%%%%%%%%%%%%%%%%%%%%%%%%%%%%%%%%%%%%%%%%%%%%%%%%%%%%%%%%%%%%%%%%%%%%%%%%%%%%%%%%%
%
%   Filename    : abstract.tex 
%
%   Description : This file will contain your abstract.
%                 
%%%%%%%%%%%%%%%%%%%%%%%%%%%%%%%%%%%%%%%%%%%%%%%%%%%%%%%%%%%%%%%%%%%%%%%%%%%%%%%%%%%%%%%%%%%%%%%%%%%%%%

\begin{abstract}
From 150 to 200 words of short, direct and complete sentences, the abstract 
should be informative enough to serve as a substitute for reading the thesis document 
itself.  It states the rationale and the objectives of the research.  

In the final thesis document (i.e., the document you'll submit for your final thesis defense), the 
abstract should also contain a description of your research results, findings, 
and contribution(s).

%
%  Do not put citations or quotes in the abract.
%

Keywords can be found at \url{http://www.acm.org/about/class/class/2012?pageIndex=0}.  Click the 
link ``HTML'' in the paragraph that starts with ''The \textbf{full CCS classification tree}...''.

\begin{flushleft}
\begin{tabular}{lp{4.25in}}
\hspace{-0.5em}\textbf{Keywords:}\hspace{0.25em} & Keyword 1, keyword 2, keyword 3, keyword 4, etc.\\
\end{tabular}
\end{flushleft}
\end{abstract}
